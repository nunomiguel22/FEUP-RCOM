\documentclass[11pt]{article}
\usepackage[utf8]{inputenc}
\usepackage{graphicx}
\usepackage{titling}
\usepackage{index}
\usepackage{fancyhdr}
\usepackage{enumitem}
\usepackage{microtype}
\usepackage{blindtext}
\usepackage{amsmath}
\usepackage{wrapfig}


\pretitle{
\begin{center}
\vspace{2.5cm}

\includegraphics[scale=0.2]{FEUP_LOGO.png}

\vspace{2cm}

\LARGE \textbf{Redes de Computadores}

\vspace{0.5cm}

}
\posttitle{\end{center}}

\title{\large{\textbf{Ligação de Dados}} }
\author{Nuno Miguel Fernandes Marques - 201708997 - MIEIC}
\date{\today}

\begin{document}
\maketitle
\thispagestyle{empty}

\newpage
\thispagestyle{fancy}
\fancyhf{}
\fancyhead[R]{asd}
\fancyfoot[R]{\thepage}
\renewcommand*{\footrulewidth}{1pt}

\section*{Sumário}
Este relatório e feito no âmbito do primeiro trabalho laboratorial de Redes de computadores. O trabalho consiste na transmissão de ficheiros usando a porta de série.

CONCLUSOES

\section*{Introdução}

\section*{Arquitectura}

\section*{Estrutura do código}

\section*{Casos de uso principais}

\section*{Protocolo de ligação lógica}

\section*{Protocolo de aplicação}

\section*{Validação}

\section*{Eficiência do protocolo de ligação de dados}

\section*{Conclusões}





\end{document}
